\documentclass[10pt]{beamer}

\usepackage{beamerthemesplit}
\usepackage{ucs}
\usepackage[utf8]{inputenc}
\usepackage{hyperref}
\usepackage{graphicx}
\usepackage[french]{babel}
\usepackage{listings}

\usetheme{Dresden}
\setbeamercolor{structure}{fg=white!50!black}

\newcommand<>{\rtgicolor}[1]{{\color#2[rgb]{1,0.6,0}#1}}
\newenvironment{rtgicolorenv}{\only{\color[rgb]{1,0.6,0}}}{}

\title{Captation de données web}
\author{Camille Maussang}
\institute{camille.maussang@rtgi.fr\\ RTGI}
\date{\today}

\begin{document}

%\section{Introduction}

\logo{\includegraphics[width=1cm]{logo}}

\frame{
	\titlepage
}

\frame{
  \frametitle{Qui suis-je ?}

	\begin{itemize}[<+-| rtgicolor@+>]
	\item Camille Maussang (\texttt{cmaussan})
	\item RTGI fabrique des outils d'analyse du web social
	\end{itemize}
}

\section{Le web}

\frame{
	\frametitle{Qu'est-ce que le web et comment le saisir ?}

	\begin{block}<+->{\rtgicolor<.>{Le web est un corpus de documents}}
	\begin{itemize}[<+-| rtgicolor@+>]
	\item ouvert,
	\item hétérogène,
	\item et dynamique.
	\end{itemize}
	\end{block}

%	\pause
}

\frame{
	\frametitle{Qu'est-ce que le web et comment le saisir ?}

	\begin{block}<+->{\rtgicolor<.>{Le web peut être représenté par des graphes}}
	\begin{itemize}[<+-| rtgicolor@+>]
	\item où les noeuds sont :
		\begin{itemize}[<+-| rtgicolor@+>]
		\item des pages,
		\item des sites,
		\item des mots,
		\item ou des gens,
		\end{itemize}
	\item et les arcs des liens.
	\end{itemize}
	\end{block}

}

\frame{
	\frametitle{Qu'est-ce que le web et comment le saisir ?}

	\begin{block}<+->{\rtgicolor<.>{Capter des données sur le web requiert un certain nombre de ressources}}
	\begin{itemize}[<+-| rtgicolor@+>]
	\item Bande passante
	\item Stockage
	\item Temps machine
	\end{itemize}
	\end{block}
}

\frame{
	\frametitle{Qu'est-ce que le web et comment le saisir ?}

	\begin{block}<+->{\rtgicolor<.>{Donc :}}
	\begin{itemize}[<+-| rtgicolor@+>]
	\item Que cherchons-nous ?
	\item Que faire pour récupérer ce qui nous est important ?
	\item Toujours penser "heuristiques" et "effets de bord" !
	\end{itemize}
	\end{block}
}

\frame{
	\frametitle{Qu'est-ce que le web et comment le saisir ?}

	\begin{alertblock}<+->{Ne jamais oublier !}
	\uncover<+->{\rtgicolor<.>{Le web c'est}}
	\uncover<+->{\rtgicolor<.>{n'importe qui (ouvert)}}
	\uncover<+->{\rtgicolor<.>{qui publie n'importe quoi n'importe comment (hétérogène)}}
	\uncover<+->{\rtgicolor<.>{n'importe quand (dynamique).}}
	\end{alertblock}
}

\section{Crawler}

\frame{
	\frametitle{Définitions}

	\begin{block}<+->{\rtgicolor<.>{Normes, recommandations et standards}}
	\begin{itemize}[<+-| rtgicolor@+>]
	\item Norme (ISO/RFC) : \texttt{HTTP}, \texttt{URL}, \texttt{SGML}, \texttt{HTML} 1-2, \texttt{MIME}
	\item Recommandation W3C : \texttt{HTML} 3-4-5, \texttt{XHTML} 1, \texttt{CSS}, \texttt{DOM}
	\item Standards : \texttt{PDF} et Flash (Taux de pénétration $> 99\%$) 
	\end{itemize}
	\end{block}

	\begin{block}<+->{\rtgicolor<.>{Web dynamique}}
	\begin{itemize}[<+-| rtgicolor@+>]
	\item \emph{server-side} : \texttt{CGI}, \texttt{PHP}, \texttt{ASP}, \texttt{JSP}
	\item \emph{client-side} : \texttt{Javscript}, Flash, ActiveX
	\end{itemize}
	\end{block}

}

\begin{frame}[fragile]
\frametitle{Prologue}

\begin{block}<+->{\rtgicolor<.>{Principe}}
\begin{itemize}[<+-| rtgicolor@+>]
\item Télécharger \emph{une} page
\end{itemize}
\end{block}

\lstset{
	language=bash,
	basicstyle=\tiny\ttfamily,
	keywordstyle=\color[rgb]{1,0.6,0},
%	moredelim=[s][\color{gray}]{/}{/},
	stringstyle=\color{gray},
	showstringspaces=false
}
\begin{uncoverenv}<+->
\begin{lstlisting}
$ wget 'http:://www.example.org/' -O page.html
$ curl 'http:://www.example.org/' > page.html
$ perl -MLWP::Simple -e 'print get("http://www.example.org/")' > page.html
\end{lstlisting}
\end{uncoverenv}

\begin{block}<+->{\rtgicolor<.>{Déjà des problèmes}}
\begin{itemize}[<+-| rtgicolor@+>]
\item Type de fichier
\item Encodage
\item Contenu (\texttt{HTML})
\end{itemize}
\end{block}

\end{frame}

\frame{
	\frametitle{Crawler}

	\uncover<+->{\rtgicolor<.>{Principe}}
	\begin{itemize}[<+-| rtgicolor@+>]
	\item Télécharger 1 page
	\item Extraire les liens
	\item Télécharger les pages pointées par les liens
	\item etc. etc.
	\end{itemize}

}

\begin{frame}[fragile]
\frametitle{Crawler}
\lstset{
	language=Perl,
	basicstyle=\tiny\ttfamily,
	keywordstyle=\color[rgb]{1,0.6,0},
	moredelim=[s][\color{gray}]{/}{/},
	stringstyle=\color{gray},
	showstringspaces=false,
    numbers=left
}
\begin{lstlisting}
use strict; use warnings;
use LWP::Simple;

my ( $max_depth, @seed ) = @ARGV or die( 'need depth and url(s)' );
my @already_visited = ();
my $depth = 0;
my @to_visit = @seed;

while( $depth <= $max_depth && @to_visit ) {
    print "crawling depth $depth\n";
    my @links = ();
    for my $url ( @to_visit ) {
        if( my $content = get( $url ) ) {
            while ( $content =~ m/<a href="([^"]+)"/gi) { push @links, $1 }
        }
        push @already_visited, $url;
        print "$url visited.\n";
    }
    @to_visit = ();
    for my $url_to_check ( @links ) {
        my $to_push = 0;
        for my $url_visited ( @already_visited ) {
            if( $url_to_check eq $url_visited ) { $to_push = 0; last; }
            $to_push = 1;
        }
        push @to_visit, $url_to_check
            if( $to_push && !grep{ $_ eq $url_to_check } @to_visit );				
    }
    $depth++;
}
print "end.\n";
\end{lstlisting}
\end{frame}

\frame{
	\frametitle{Crawler}

	\begin{itemize}[<+-| rtgicolor@+>]
	\item Métriques (distance, profondeur, etc.)
	\item Performance (goulots d'étranglement)
	\item Scalabilité
	\end{itemize}

}

\frame{
	\frametitle{Crawling}

	\begin{block}<+->{\rtgicolor<.>{De nouveaux problèmes}}
	\begin{itemize}[<+-| rtgicolor@+>]
	\item Politesse
		\begin{itemize}[<+-| rtgicolor@+>]
		\item DoS (\emph{Denial of Service}) : DNS, Serveurs HTTP
		\item Blacklistage officiel (\texttt{robots.txt}, \texttt{sitemap.xml}, etc.)
		\item Blacklistage officieux (\emph{cloaking}, pièges à robot)
		\end{itemize}
	\item Addressage
		\begin{itemize}[<+-| rtgicolor@+>]
		\item Normalisation d'\texttt{URL} (doublons)
       	\item Site ou page ? 
        \item Plusieurs permaliens pour un seul contenu (GYM aide un peu)
		\end{itemize}
	\item Autres...
		\begin{itemize}[<+-| rtgicolor@+>]
		\item Javascript
       	\item Web privé
		\end{itemize}
	\end{itemize}
    \end{block}
}

\frame{
	\frametitle{Crawler}
		
	\begin{block}<+->{\rtgicolor<.>{Astuces}}
	\begin{itemize}[<+-| rtgicolor@+>]
	\item Utiliser les \texttt{headers HTTP}
    \item User-agent et \texttt{random}
	\end{itemize}
    \end{block}

	\begin{block}<+->{\rtgicolor<.>{Principes du \emph{Focused crawler}}}
	\begin{itemize}[<+-| rtgicolor@+>]
	\item Ne télécharger que les pages pertinentes
	\end{itemize}
	\begin{itemize}[<+-| rtgicolor@+>]
	\item Indicateurs topologiques
	\item Indicateurs sémantiques
	\end{itemize}
    \end{block}
}

\section{Autres méthodes}

\frame{
	\frametitle{Aggrégation}

	\begin{block}<+->{Principe}
	\uncover<+->{\rtgicolor<.>{Syndication ou comment "retourner" l'accès aux données}}
	\end{block}

	\begin{block}<+->{\rtgicolor<.>{Problèmes}}
	\begin{itemize}[<+-| rtgicolor@+>]
	\item Atom, \texttt{RSS}, encore mille versions
	\item Flux complet / partiel / vide avec ou sans \texttt{HTML} 
	\end{itemize}
	\end{block}
}

\frame{
	\frametitle{Scraping}

	\begin{block}<+->{Principe}
	\uncover<+->{\rtgicolor<.>{Analyser une page web pour en extraire une information spécifique}}
	\end{block}

	\begin{block}<+->{\rtgicolor<.>{Problèmes}}
	\begin{itemize}[<+-| rtgicolor@+>]
	\item \texttt{DOM} ou Regexp ou les deux
	\item Template et dynamisme des pages scrapées 
	\end{itemize}
	\end{block}
}

\frame{
	\frametitle{API}

	\begin{block}<+->{Principe}
	\uncover<+->{\rtgicolor<.>{Utiliser les API de certains sites pour collecter la donnée}}
	\end{block}

	\begin{block}<+->{\rtgicolor<.>{Problèmes}}
	\begin{itemize}[<+-| rtgicolor@+>]
	\item Limitations
	\item API propriétaires 
	\end{itemize}
	\end{block}
}

\section{Exemple}

\frame{
	\frametitle{Un exemple concret}

}

%Problèmes concrets:
%  - Constituer un corpus en scrapant un moteur de recherche

\frame{
	\frametitle{Wikipédia est ton ami :)}

	\begin{itemize}
	\item HTML : \url{http://en.wikipedia.org/wiki/HTML}
	\item Crawler : \url{http://en.wikipedia.org/wiki/Web_crawler}
	\item Focused Crawler : \url{http://en.wikipedia.org/wiki/Focused_crawler}
	\item Scraping : \url{http://en.wikipedia.org/wiki/Web_scraping}
	\item Cloaking : \url{http://en.wikipedia.org/wiki/Cloaking}
	\item DoS : \url{http://en.wikipedia.org/wiki/Denial-of-service_attack}
	\end{itemize}
}

\frame{
	\frametitle{Merci!}	

    \begin{itemize}
    \item \url{http://labs.rtgi.eu/}
    \item \url{http://github.com/cmaussan/Picrowler}
    \end{itemize}

    \pause
}

\end{document}
